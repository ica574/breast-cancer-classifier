\documentclass[main]{subfiles}

\begin{document}
\chapter{Introduction}
\label{chap:introduction}
Breast cancer remains one of the leading causes of cancer mortality among women around the world. Routine screening processes can prompt early detection of cancer and significantly improve patient outcomes, yet they are time consuming, and success rates vary depending on the observer. Across the world, many hospitals are participating in large-scale research programs to investigate the feasibility of using deep learning systems, such as convolutional neural networks (CNNs), to match or exceed the diagnostic experience of human radiologists in the detection of breast cancer.

Based on the premise of this initiative, project 3.2 titled \textemdash Deep Learning Breast Cancer Detection explores the potential of applying CNN-based models to publicly available data sets such as the Digital Database for Screening Mammography (DDSM), as well as other technologies such as vision transformers, and transfer learning. The objective of this project is two-fold: to develop a working proof-of-concept that achieves statistical significance in its results on the DDSM data set, and to evaluate the integration of such a prototype into existing clinical workflows, identifying potential challenges and limitations.

This introduction begins by stating the clinical, and technical motivation for this project, then presents research questions, and objectives, as well as an outline of the report structure.

\section{User Requirements}
\label{sec:intro_user_requirements}
The main users of this system would be radiologists, and clinicians within a clinical setting such as the NHS, particularly involved in the detection of breast cancer. Radiologists require a system that can provide accurate malignancy predictions that prioritises high-risk cases to reduce the workload, and offers explainable outputs (like through Grad-CAM heatmaps) to foster trust in the model's decisions. Clinicians require a system that can seamlessly integrate into existing systems, and supports rapid inference time, whilst minimising false positives that can lead to patient anxiety, and unnecessary follow-ups. As indirect stakeholders, patients would benefit from a system that can improve the precision of breast cancer screening, leading to earlier detection, and treatment, ultimately improving patient outcomes.

\section{Stakeholder Requirements}
\label{sec:intro_stakeholder_requirements}
This project aligns with the NHS 2025 initiative to explore AI-assisted mammoraphy systems, that aim to replace two radiologists in the screening process (as detailed in the project brief). Stakeholders, therefore, include NHS radiologists, hospital IT departments (to integrate the system into existing workflows), and regulatory bodies (such as the MHRA for clinical approval). The system must comply with medical device regulations, whilst ensuring data privacy in accordance with GDPR regulations. It must also be computationally feasible on hospital hardware through GPU-enabled servers. By understanding these needs, the system aims to reduce diagnostic, and turnaround times, alleviate radiologist workload, and improve patient outcomes through more accurate breast cancer screening.

\section{Motivation}
\label{sec:intro_motivation}
Mammography is widely known as the de facto standard for early detection of breast cancer. Despite this, the visual similarity between benign, and malignant lesions can challenge even the most experienced of radiologists, leading to skewed results, and misdiagnosis. Often, false positives lead to unnecessary follow-ups, and false negatives result in delayed treatment, which can prove fatal. Deep learning can significantly reduce these problems by learning from large collections of labelled scans, complementing radiologists, and improving the accuracy of mammography.

Convolutional neural networks (CNNs) can be used to automatically extract features from an image, including edges, and complex textures, that can prove demonstrably effective at weeding out subtle patterns in pathology. Various studies have demonstrated the prowess of CNNs in tasks such as tumour segmentation, and classification on medical imaging modalities such as CT, and MRI scans, however, applications to mammography remain relatively untested, in part due to the challenges such as standardisation, and generalisation.

The DDSM data set is a large-scale collection of digital mammograms, annotated by radiologists for the presence or absence of malignant lesions. It is widely used in research, allowing for effective benchmarking of model performance against established studies, and comparing CNN outputs to ground-truth labels. In addition, the use of the DDSM ensures that the prototype is developed on a data set that is representative of real-world clinical data, and is reproducible, allowing for further extensions, and collaborations.

In addition to technical performance, an automated system must be understood in terms of how it could enhance clinical workflows. Thus, a CNN that is reliable enough could be used as an assistant to radiologists, prioritising high-risk cases for human review, and reducing the burden of radiologist workload, and turnaround times. On the other hand, understanding how the system can fail, such as misclassification issues, could guide refinements in data preprocessing pipelines, model architecture, and evaluation metrics.

\section{Research Questions, and Objectives}
\label{sec:intro_research_questions}
This project consists of the following central research questions.

\begin{enumerate}
    \label{rq:intro}
    \item \textbf{Model Feasibility} \textemdash\ Can a CNN trained on the DDSM data set achieve classification metrics (like AUC, sensitivity, specificity) that approach or exceed those for human readers under similar conditions? \label{rq:model_feasibility}
    \item \textbf{Prototype Evaluation} \textemdash\ What do preliminary results demonstrate about the feasibility of integrating a CNN-based system into clinical workflows? How might these insights inform augmentations to model architecture or data augmentation methods to improve the system? \label{rq:prototype_evaluation}
    \item \textbf{Clinical Integration} \textemdash\ What are some of the practical considerations, like inference time, interpretability, and failures that would affect the adoption of such deep learning systems in a clinical setting? \label{rq:clinical_integration}
\end{enumerate}

In order to answer these questions, the following will be performed.

\begin{itemize}
	\item Develop a CNN architecture using TensorFlow, and Keras, optimised for binary classification of patches, using practices like data set splitting, transfer learning, and regularisation.
	\item Train the model on the DDSM data set, and report performance on a test partition using metrics like accuracy, sensitivity, specificity, and AUC to determine diagnostic accuracy.
	\item Analyse the model's performance to identify the categories of issues where performance degrades, and identify mitigation strategies.
	\item Reflect on integration into clinical settings, discussing how such a system helps to reduce radiologist workload in reading mammograms, and error rates.
\end{itemize}

\noindent Note: Due to time constraints, the previously mentioned Django web application was omitted. Instead, the focus is on developing a proper convolutional neural network that is able to accurately discriminate between benign and malignant lesions in mammograms, and demonstrate its decisions through the Grad-CAM framework.

\section{Report Structure}
\label{sec:intro_report_structure}
\begin{itemize}
	\item \textbf{Chapter 1} \textemdash\ Introduction \textemdash\ This chapter introduces the motivation for the project, and outlines the research questions, and objectives.
	\item \textbf{Chapter 2} \textemdash\ Literature Review \textemdash\ This chapter reviews the literature on breast cancer screening, and applications of deep learning applications to mammography.
	\item \textbf{Chapter 3} \textemdash\ Design \textemdash\ This chapter describes the design of the CNN architecture, and the data preprocessing pipeline.
	\item \textbf{Chapter 4} \textemdash\ Implementation \textemdash\ This chapter describes the manner in which the system was implemented, including a description of the implemented model, and the way it was trained.
	\item \textbf{Chapter 5} \textemdash\ Evaluation \textemdash\ This chapter recites the evaluation pipelines, and the attained scores from the model defined in the previous chapter.
	\item \textbf{Chapter 6} \textemdash\ Conclusion \textemdash\ The final chapter summarises the findings of the project, and discusses the implications of the results, as well as future work.
\end{itemize}


\noindent Note: This project borrows inspiration from \textit{https://github.com/dustoff06/BreastCancers}.

\end{document}
