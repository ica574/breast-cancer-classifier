\documentclass[../main]{subfiles}

\begin{document}
\chapter{Conclusion}
This project involves an end-to-end image classification pipeline for distinguishing two classes of mammographic scans using deep learning. Transfer learning is applied, with a pre-trained convolutional neural network being used for the classification of scans. The system ingests data from TFRecord files, originating from the DDSM dataset. These are then preprocessed for the model to use as input. The latter is then trained, and evaluated on this data. This approach allows for the efficient use of pre-trained models, which can significantly reduce the time and resources required for training a model from scratch. In this pursuit, this project has been successful, and may be further extended to meet the exigencies of a clinical setting.

\section{Future Work}
This project demonstrates a solid proof-of-concept, and a foundation for future work. The final system will require further refinement, and optimisation, as well as additional features, and functionality to make it more user-friendly, and robust.

\begin{itemize}
	\item \textbf{Model Optimisation} \textemdash\ The model can be further optimised by adjusting its architecture, and increasing the number of epochs.
	\item \textbf{Advanced Architectures} \textemdash\ Alternative architectures could be experimented with like ResNet50, and EfficientNet that may offer better performance.
	\item \textbf{Data Augmentation} \textemdash\ tf.image operations like random flips, rotations, and brightness adjustments could be introduced to improve generalisation.
	\item \textbf{User Interface} \textemdash\ A user-friendly interface can be developed to allow users to easily upload images, and view results.
	\item \textbf{Deployment} \textemdash\ The model can be deployed as a web service through REST APIs, and be inferenced through them for user-friendly accessibility, particularly for clinical settings.
	\item \textbf{CI/CD} \textemdash\ Implementing continuous integration, and continuous deployment (CI/CD) practices such as the introduction of containerisation with Docker will ensure that the system is easily maintainable, scalable, and transferable across different platforms.
\end{itemize}

\end{document}
